\documentclass[11pt, a4paper]{article}

% --- UNIVERSAL PREAMBLE BLOCK ---
\usepackage[a4paper, top=2.5cm, bottom=2.5cm, left=2.5cm, right=2.5cm]{geometry}
\usepackage{fontspec}

\usepackage[spanish, bidi=basic, provide=*]{babel}

\babelprovide[import, onchar=ids fonts]{spanish}
\babelprovide[import, onchar=ids fonts]{english}

% Set default/Latin font to Sans Serif in the main (rm) slot
\babelfont{rm}{Noto Sans}

% Add because main language is not English
\usepackage{enumitem}
\setlist[itemize]{label=-}

% --- ADDITIONAL PACKAGES FOR THIS TEMPLATE ---
\usepackage{tcolorbox} % Para la caja de "Nota Importante"
\usepackage{titlesec}  % Para formato de títulos
\usepackage{hyperref}  % Para enlaces
\usepackage{xcolor}    % Para colores

% Configuración de colores
\definecolor{warningbg}{RGB}{255, 240, 220}
\definecolor{warningborder}{RGB}{255, 150, 0}
\definecolor{sectionblue}{RGB}{0, 70, 150}

% Título del documento
\title{\textbf{Guía de Diseño de la Solución y Arquitectura del Sistema}\\ \large Human Perception in Computer Vision (HPCV)}

\author{Prof. Pablo F. Ordoñez Ordoñez}
\date{}

\begin{document}

\maketitle

% ⚠️ NOTA IMPORTANTE
\begin{tcolorbox}[colback=warningbg, colframe=warningborder, title=\textbf{⚠️ NOTA IMPORTANTE - LEER ANTES DE INICIAR}]
Para proceder con esta fase de Diseño, es \textbf{obligatorio} haber incorporado todas las correcciones y observaciones recibidas durante la revisión de la fase anterior (Análisis del Problema). Este documento debe reflejar una versión madura de la definición del problema antes de detallar la solución técnica.
\end{tcolorbox}

\vspace{0.5cm}

\section{Título del Proyecto (Refinado)}
Debe reflejar la integración de Percepción Humana, Visión por Computador e IA.
\begin{itemize}
    \item \textit{Ejemplo: "Sistema de detección de fatiga visual (HP) mediante análisis de parpadeo y PERCLOS (CV) utilizando Redes Neuronales Recurrentes (AI)."}
\end{itemize}

\section{Datos del Estudiante}
\begin{itemize}
    \item \textbf{Nombre completo:} 
    \item \textbf{Correo institucional:} 
    \item \textbf{Carrera y ciclo:} 
\end{itemize}

\section{Definición del Problema (Versión Final Corregida)}
Resumen ejecutivo (máx. 1 página) integrando las correcciones de la Fase 1.
\begin{itemize}
    \item \textbf{3.1. Contexto:} Escenario donde ocurre el problema (educativo, médico, industrial).
    \item \textbf{3.2. Dimensión Human Perception (HP):} Procesos cognitivos/visuales afectados (atención, carga cognitiva, sesgos, FATE).
    \item \textbf{3.3. Dimensión Computer Vision (CV):} Necesidad de extracción automatizada de características visuales.
    \item \textbf{3.4. Dimensión AI/ML/DL:} Justificación de modelos inteligentes para patrones complejos.
\end{itemize}

\section{Objetivos (Refinados)}
\begin{itemize}
    \item \textbf{4.1. Objetivo General:} (Integrando HP + CV + IA).
    \item \textbf{4.2. Objetivos Específicos:} (Técnicos, medibles y orientados a la implementación).
\end{itemize}

\section{Modelos de Análisis (Versión Corregida)}
\textit{\textbf{Instrucción:} Presente aquí los diagramas de la Fase 1 \textbf{ya corregidos}. Esta es la base lógica del sistema antes de entrar en la arquitectura técnica.}

\begin{itemize}
    \item \textbf{6.1. Modelo de Contexto:} Relación entre Usuario, Sensores/Cámaras y Entorno.
    \item \textbf{6.2. Diagrama de Casos de Uso (HP/CV):} Interacciones centradas en la percepción y la captura de datos. Diagrama de casos de uso del sistema, por cada Cu dos escenarios con su respectivo workflow.
    \item \textbf{6.3. Modelo Conceptual de Dominio:} Entidades clave (ej. \texttt{Estímulo}, \texttt{FrameVideo}, \texttt{FeatureMap}, \texttt{Predicción}).
    \item \textbf{6.4. Diagrama de Flujo del Fenómeno Perceptual:} Secuencia de acciones humanas y puntos de intervención del sistema.
\end{itemize}

\section{Diseño de la Solución (Arquitectura y Modelado Formal)}
\textit{\textbf{Instrucción General:} Utilice un lenguaje de modelado formal (UML, SysML, C4, Diagramas de Bloques) para describir la estructura técnica. Tiene libertad para elegir los diagramas que mejor expliquen su solución.}

\subsection{Arquitectura General del Sistema (Pipeline de Visión)}
Diagrama de alto nivel que muestre el flujo completo de la información.
\begin{itemize}
    \item \textit{Debe evidenciar:} Adquisición (Cámaras/Sensores) $\rightarrow$ Preprocesamiento $\rightarrow$ Módulo IA $\rightarrow$ Interfaz/Feedback.
\end{itemize}

\subsection{Diagrama de Secuencia General}
Diagrama que ilustre la cronología de una interacción completa (Happy Path).
\begin{itemize}
    \item \textit{Ejemplo:} Estímulo Visual presentado $\rightarrow$ Ojo humano reacciona $\rightarrow$ Cámara captura frame $\rightarrow$ CV detecta ROI $\rightarrow$ Modelo IA clasifica $\rightarrow$ Sistema guarda métrica.
\end{itemize}

\subsection{Diseño del Componente de Percepción Humana (HP)}
Diseño de la interacción y los estímulos.
\begin{itemize}
    \item \textbf{Diseño de Estímulos Visuales:} ¿Qué verá el usuario? (Imágenes, videos, UI experimental). Incluir bocetos o parámetros (color, contraste, frecuencia).
    \item \textbf{Protocolo de Interacción:} ¿Cómo responde el usuario? (Gestos, mirada, teclado, voz).
    \item \textbf{Consideraciones Éticas (FATE):} Diseño para mitigar sesgos o fatiga visual excesiva.
\end{itemize}

\subsection{Diseño del Componente CV e IA (El Núcleo Técnico)}
Especificación detallada de los algoritmos y modelos.
\begin{itemize}
    \item \textbf{Estrategia de Adquisición:} Especificaciones de hardware (Resolución, FPS, iluminación controlada).
    \item \textbf{Pipeline de Preprocesamiento (CV):} Diagrama de flujo de las operaciones clásicas (Grayscale, Histogram Eq, Data Augmentation, ROI Extraction).
    \item \textbf{Arquitectura del Modelo (AI/DL):} Diagrama de la estructura interna del modelo seleccionado.
    \begin{itemize}
        \item \textit{Si es CNN:} Capas convolucionales, pooling, dense layers.
        \item \textit{Si es Transformer/RNN:} Bloques de atención, secuencias.
        \item \textit{Entradas y Salidas:} Definición de tensores (ej. Input: $224\times224\times3$, Output: Vector de probabilidades).
    \end{itemize}
\end{itemize}

\section{Resultados Preliminares de Prototipado}
\textit{\textbf{Instrucción:} Evidencia visual de que la implementación ha comenzado y es viable.}

\subsection{Dataset y Datos de Entrada}
Muestras de los datos que alimentarán al sistema.
\begin{itemize}
    \item Capturas de imágenes/video del dataset (propio o público).
    \item Ejemplos de etiquetado (Ground Truth) si aplica.
\end{itemize}

\subsection{Pruebas de Visión por Computador (CV) (Opcional)}
Resultados iniciales de procesamiento sin IA o pre-procesamiento.
\begin{itemize}
    \item Ejemplos: Detección de rostros (bounding boxes), filtros de bordes, segmentación por color, umbralización.
\end{itemize}

\subsection{Prototipo de Interfaz o Estímulos}
\begin{itemize}
    \item Capturas de la GUI/bocetos/Wireframes donde se presentarán los estímulos o se mostrarán los resultados al usuario.
\end{itemize}

\section{Plan de Implementación Técnica}
Detalle las herramientas definitivas:
\begin{itemize}
    \item \textbf{Lenguajes:} (ej. Python, C++).
    \item \textbf{Librerías CV:} (ej. OpenCV, Scikit-Image, PIL).
    \item \textbf{Frameworks DL/ML:} (ej. PyTorch, TensorFlow/Keras, YOLOv8, MediaPipe).
    \item \textbf{Hardware:} (ej. Webcam, GPU NVIDIA, Jetson Nano, Raspberry Pi).
\end{itemize}

\section{Estrategia de Evaluación}
\begin{itemize}
    \item \textbf{Métricas de IA/CV:} (Accuracy, Precision, Recall, F1-Score, IoU, FPS, Curvas de Loss).
    \item \textbf{Métricas de Percepción Humana:} (Tiempos de reacción, tasa de error humano, cuestionarios subjetivos).
    \item \textbf{Plan de Validación:} ¿Cómo se dividirá el dataset (Train/Val/Test)? ¿Se harán pruebas con usuarios reales?
\end{itemize}

\section{Referencias}
Formato IEEE.

\section{Anexos}
\begin{itemize}
    \item Prompt Engineering (Registro de uso de IA generativa).
    \item Enlaces a repositorios/bocetos/wireframes/simulaciones de código o datasets.
\end{itemize}

\end{document}